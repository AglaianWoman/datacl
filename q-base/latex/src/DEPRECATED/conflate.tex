\startreport{Conflation Analyzer}
\reportauthor{Ramesh Subramonian}

\section{Introduction}

The overall goal of this analysis is to 
\be
\item make it easy to assess the quality of title standardization in an
ongoing fashion
\item suggest when and where dictionaries need to be updated
\item measure quality when the data is sliced in a particular way e.g.,
  by geography
\item find titles that are mapped too broadly e.g., {\tt engineer} to
  {\tt engineeer} versus {\tt engineeer} to {\tt hardware engineer}.
\ee

The baby step described here is the Conflation Analyzer. 
It helps find multiple raw titles that map to the same clean title.

\section{Set Up}

The algorithm consists of the following steps
\be
\item Load clean titles --- Section~\ref{load_clean_titles}
\item Create clean titles for a number of raw titles ---
Section~\ref{load_and_scrub_raw_titles}
\ee


\subsection{Load Clean Titles}
\label{load_clean_titles}

\bi
\item \verb+cd ~/WORK/Q/src/CONFLATE_ANALYZER+
\item \verb+sh load_clean_title_dict.sh final clean titles.txt+
\ei

\subsection{Load and Scrub Raw Titles}
\label{load_and_scrub_raw_titles}

The input to this routine is an extract from a data warehouse dump. It
is a tab separated file, the 4th line of which is the raw title. 
The output of this routine is a file containing the clean title
associated with the raw title. If more than one title is provided, the
first is extracted. This limitation will be removed in subsequent
versions \TBC

\bi
\item \verb+cd ~/WORK/TITLE/RON/+
\item 
\verb+cut -f 4 <mbr_pos.txt> | php prep_for_ron.php _just_titles.txt _title_count.txt N+, 
  where \(N\) is the number of lines
we want to limit our experiment to. Use -1 for no limit.
\item \verb+java -cp . TitleMapping/TitleMapping _title_count.txt _ron_op.txt+
\item \verb+php first_clean_title.php _ron_op.txt _raw.txt _clean.txt+
\ei

\section{Queries}
\subsection{Conflation Counter}
\label{conflation_counter}

The input is a clean title. The output is a set of pairs \(\{rtitle_i,
count_i\}\) which lists the raw titles that map to this clean title
and the number of times that each of these raw titles occurs.

\subsection{Raw Titles with No Clean Title}
\label{no_clean_title}
The output is a set of pairs \(\{rtitle_i, count_i\}\) which lists the raw titles 
that do not map to any clean title and how many times each such raw
title occurs.

\subsection{Most Conflated}
\label{most_conflated}
The aim of this analysis is to find clean titles which are ``most
conflated''. Intuitively, this means that if there are many diferent raw
titles that map to a given clean title, then that clean title becomes a
suspect. It does not mean that it is a wrong mapping! It is merely an
indicator of something worthy of further investigation.

We now proceed more formally. We define the following tables
\be
\item \(T_{RC}\). Columns are
\be
\item \(C\) --- clean title 
\item \(R\) --- raw title 
\item \(n_R\) --- number of times raw title occurred
\ee
\item \(T_C\). Columns are
\be
\item \(C\) --- clean title 
\ee
\ee
Rough values for some of the parameters listed below are
\be
\item \(\alpha\) --- 0.05
\item \(\beta_1\) --- 100
\item \(\beta_2\) --- 10
\ee
Algorithm is as follows
\be
\item Create field \(one\) in \(T_{RC}\) such that \(\forall i: one[i] = 1\)
\item Create field \(n_C\) in \(T_C\) such that \(T_C[C=c'].n_C = \sum
T_{RC}(C=c').one\). 
\item Create \(\alpha n_C\) in \(T_C\) as \(\alpha \times n_C\)
\item Create \(\alpha n_C\) in \(T_{RC}\) using \(C\) as link field
\item Create \(x_1\) in \(T_{RC}\) such that \(n_R > \alpha n_C\)
\item Create \(x_2\) in \(T_{RC}\) such that \(n_R > \beta_2\)
\item Create \(x\) in \(T_{RC}\) such that \(x \la x_1 \wedge x_2\)
\item Create \(y\) in \(T_{RC}\) such that \(n_R > \beta_1\)
\item Create \(z\) in \(T_{RC}\) such that \(z \la x \vee y\)
\item Let \(T'_{RC} \la T_{RC}[z = true]\)

\item Create field \(one\) in \(T'_{RC}\) such that \(\forall i: one[i] = 1\)
\item Create field \(cnt\) in \(T_C\) such that \(T_C[C=c'].n_C = \sum
T'_{RC}(C=c').one\). 
\item Create field \(y\) in \(T'_{RC}\) such that \(cnt > 1\)
\item Join field \(y\) from \(T_C\) to \(T'_{RC}\)

\item Let \(T''_{RC} \la T'_{RC}[y = true]\)
\item Create a sort order in \(T''_{RC}\) using \(N_C\) as primary key
and \(C\) as secondary key so that more commonly occurring titles occur
first and all different mappings of raw title to a given clean title
occur together
\item Print the following fields from \(T''_{RC}\) using above sort
order --- raw title, number of times raw title occurs, clean title,
      number of times clean title occurs

\ee

\subsection{Previous titles}
\label{previous_titles}

Let us say that the results of Section~\ref{most_conflated} indicates
that the raw titles \verb+md+ and \verb+managing director+ both map to
\verb+managing director+. Let us further assume that we consider this
suspicious since we believe that the most common expansion of \verb+md+
is \verb+medical doctor+. To test our hypothesis, we would ideally look 
at the entire profile of members with the title \verb+md+. A less exact
but faster approach would be to use the next and previous
titles as an approximation to the entire profile.

Algorithm is as follows. Let \(T_M\) be a table with (at least) the 
following fields
\be
\item m --- member ID
\item c --- chronological order of title
\item t --- user supplied title
\ee
Let \(t'\) be the title in question.
\be
\item Create field  \(x\) in \(T_M\) such that \(t = t'\)
\item Create table \(T_1 \la T_M[x=true].m\)
\item Add constant boolean field \(b\) in \(T_1\) with value true.
\item Join \(b\) in \(T_1\) to \(b\) in \(T_M\) using \(m\) as link
field
\item Note that the number of rows in \(T_M\) with \(b = true\) is at
as many as the number of rows with \(x = true\)
\item Create table \(T_2 \la T_M[b=true]\)
\item Sort \(T_2\) using \(m\) as primary key and \(c\) as secondary key
\item Create field \(b_n\) in \(T_2\) by shifting \(b\) up by one
\item Create field \(b_p\) in \(T_2\) by shifting \(b\) down by one
\item Create field \(m_n\) in \(T_2\) by shifting \(b\) up by one
\item Create field \(m_p\) in \(T_2\) by shifting \(b\) down by one
\item Create boolean field \(y\) in \(T_2\) as \(
(m = m_n \wedge b_n = true) \vee (m = m_p \wedge b_p = true) \)
\item Create \(T_3 \la T_2[y=true].t\)
\item Create \(T_4\) with fields \((t, cnt)\) as histogram of \(T_3.t\)
\item Create sort order on \(T_4\) using \(cnt\) as primary key in
  descending order
\item Print \(T_4\) using sort order
\ee
