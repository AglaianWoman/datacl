% -*- latex -*-

\documentclass{report}

% -*- latex -*-


\usepackage[left=1in, right=1in]{geometry}
\usepackage{times}
\usepackage[pdftex]{graphicx}
\usepackage{fancyhdr}
\usepackage{fancybox}
\usepackage{pstricks}
\usepackage{color}
\usepackage{sectsty}
\usepackage{makeidx}
\usepackage{tabularx}
\usepackage{ifthen}
\usepackage[pdftex,colorlinks,linkcolor=blue]{hyperref}
\usepackage{boxedminipage}
%% BELOW FROM RAMESH
\usepackage{supertabular}
\usepackage{verbatim}
\usepackage{ifthen}
\usepackage{tabularx}
\usepackage{afterpage}
\usepackage{soul}
\usepackage{amssymb}
\usepackage{amsmath}
%% ABOVE FROM RAMESH

% The main configuration file


\newcommand{\docversion}{0.1}
\newcommand{\docpublishingdate}{\today}



\makeindex

%%%%%%%%%%%%%%%%%%%%%%%%%%%%%%%%%%%%%%%%%%%%%%%%%%%%%%%%%%%%%%%%%%%%%%%%%%%%%%%%%%
% Special types of chapters
\newcommand{\specialchapter}[1]{\chapter*{#1}\markboth{#1}{}\addcontentsline{toc}{chapter}{#1}}

\newcommand{\acknowledgements}{\specialchapter{Acknowledgements}}
\newcommand{\introduction}{\specialchapter{Introduction}}
\newcommand{\preface}{\specialchapter{Preface}}
  \newcommand{\prefaceauthor}[1]{{\par\itshape \raggedleft    #1 \par}}
  \newcommand{\prefacewhere}[1]{{\par\itshape \raggedleft #1 \par}}
\newcommand{\acronyms}{\specialchapter{Acronyms}}
  \newcommand{\acro}[2]{\textsf{#1} --- #2 \par}
%%%%%%%%%%%%%%%%%%%%%%%%%%%%%%%%%%%%%%%%%%%%%%%%%%%%%%%%%%%%%%%%%%%%%%%%%%%%%%%%%%


%%%%%%%%%%%%%%%%%%%%%%%%%%%%%%%%%%%%%%%%%%%%%%%%%%%%%%%%%%%%%%%%%%%%%%%%%%%%%%%%%%
% Page and headings layout
\pagestyle{fancy}

\addtolength{\headwidth}{\marginparwidth}

\newcommand{\myhf}{
  \fancyhead{}
  \renewcommand{\headrulewidth}{1pt}
  \renewcommand{\footrulewidth}{1pt}

%  \fancyhead[R]{\rightmark{}\rule{1cm}{0pt}\includegraphics{logo}}
%  \fancyhead[R]{\rightmark{}{\hspace{0.2ex} \Ovalbox{LinkedIn}}}
  \fancyhead[L]{\leftmark{}}
%  \fancyhead[C]{v. \docversion{}, \docpublishingdate{}}
  \fancyhead[C]{\docpublishingdate{}}
%  \fancyfoot[L]{LinkedIn Proprietary and Confidential}
}

\myhf
\fancypagestyle{plain}{\myhf}

\renewcommand{\chaptermark}[1]{\markboth{#1}{}}
\renewcommand{\sectionmark}[1]{\markright{#1}}

\allsectionsfont{\sffamily}
\chapterfont{\raggedleft\sffamily\Huge}
%%%%%%%%%%%%%%%%%%%%%%%%%%%%%%%%%%%%%%%%%%%%%%%%%%%%%%%%%%%%%%%%%%%%%%%%%%%%%%%%%%

%% BY RAMESH
\setcounter{secnumdepth}{4}
%% BY RAMESH


%%%%%%%%%%%%%%%%%%%%%%%%%%%%%%%%%%%%%%%%%%%%%%%%%%%%%%%%%%%%%%%%%%%%%%%%%%%%%%%%%%
% Special for e-unibus doc commands

\newcommand{\ForLater}{
\begin{center}
{\bf NOT FOR CURRENT VERSION}
\end{center}
}
\newcommand{\TBC}{\framebox{\textbf{TO BE COMPLETED}}}
\newcommand{\DISCUSS}{\Ovalbox {\bf \textcolor{red}{FOR DISCUSSION}}}
\newcommand{\Input}{\framebox{\textsf{in}}}
\newcommand{\Output}{\framebox{\textsf{out}}}
\newcommand{\debug}[1]{\textbf{debug start} #1 \textbf{debug finish}}
\newcommand{\inx}[1]{\emph{#1}}
\newtheorem{notation}{Notation}
\newtheorem{definition}{Definition}
\newtheorem{problem_statement}{Problem Statement}
\newtheorem{invariant}{Invariant}
\newtheorem{resource_string}{Resource String}
\newtheorem{testcase}{Test Case}
\newtheorem{specification}{Specification}
\newtheorem{caution}{Caution}
\newtheorem{prereq}{Pre-requisite}
\newtheorem{action}{Action}
\newtheorem{query}{Query}
\newcommand{\be}{\begin{enumerate}}
\newcommand{\ee}{\end{enumerate}}
\newcommand{\bi}{\begin{itemize}}
\newcommand{\ei}{\end{itemize}}
\newcommand{\bv}{\begin{verbatim}}
\newcommand{\ev}{\end{verbatim}}
\newcommand{\bd}{\begin{description}}
\newcommand{\ed}{\end{description}}
\newcommand{\bpre}{\begin{prereq}}
\newcommand{\epre}{\end{prereq}}
\newcommand{\bact}{\begin{action}}
\newcommand{\eact}{\end{action}}
\newcommand{\bs}{\begin{specification}}
\newcommand{\es}{\end{specification}}
\newcommand{\btc}{\begin{testcase}}
\newcommand{\etc}{\end{testcase}}
\newcommand{\bc}{\begin{caution}}
\newcommand{\ec}{\end{caution}}
\newcommand{\la}{\leftarrow}
\newcommand{\IpArgs}{\subsection{Input Arguments}}
\newcommand{\PreReqs}{\subsection{Pre-requisities}}
\newcommand{\Actions}{\subsection{Actions}}
\newcommand{\Coverage}{{\bf To test coverage.}}

%%%%%%%%%%%%%%%%%%%%%%%%%%%%%%%%%%%%%%%%%%%%%%%%%%%%%%%%%%%%%%%%%%%%%%%%%%%


\newtheorem{theorem}{Theorem}[section]
\newtheorem{lemma}[theorem]{Lemma}
\newtheorem{proposition}[theorem]{Proposition}
\newtheorem{corollary}[theorem]{Corollary}

\newenvironment{proof}[1][Proof]{\begin{trivlist}
\item[\hskip \labelsep {\bfseries #1}]}{\end{trivlist}}
\newenvironment{intuition}[1][Intuition]{\begin{trivlist}
\item[\hskip \labelsep {\bfseries #1}]}{\end{trivlist}}
%% \newenvironment{definition}[1][Definition]{\begin{trivlist}
%% \item[\hskip \labelsep {\bfseries #1}]}{\end{trivlist}}
\newenvironment{example}[1][Example]{\begin{trivlist}
\item[\hskip \labelsep {\bfseries #1}]}{\end{trivlist}}
\newenvironment{remark}[1][Remark]{\begin{trivlist}
\item[\hskip \labelsep {\bfseries #1}]}{\end{trivlist}}

\newcommand{\qed}{\nobreak \ifvmode \relax \else
      \ifdim\lastskip<1.5em \hskip-\lastskip
      \hskip1.5em plus0em minus0.5em \fi \nobreak
      \vrule height0.75em width0.5em depth0.25em\fi}

%%%%%%%%%%%%%%%%%%%%%%%%%%%%%%%%%%%%%%%%%%%%%%%%%%%%%%%%%%%%%%%%%
% \newcommand{\Alogon}{\mbox{\fontfamily{ptm}\selectfont {\large \selectfont A} \hspace{-1.2ex} {\large \selectfont L} \hspace{-2.3ex} \raisebox{0.45ex}{ {\footnotesize \selectfont O} } \hspace{-1.80ex} {\large \selectfont G} \hspace{-1.80ex} \raisebox{-0.33ex}{ {\large \selectfont O} } \hspace{-1.8ex} {\large \selectfont N}}}




%%%%%%%%%%%%%%%%%%%%%%%%%%%%%%%%%%%%%%%%%%%%%%%%%%%%%%%%%%%%%%%%%%%%%%%%%%%%%%%%%%
% Report title page
\newcommand{\ReportAuthor}{The LinkedIn Documentation Team}
\newcommand{\reportauthor}[1]{\renewcommand{\ReportAuthor}{#1}}
\newcommand{\reporttitle}[1]{\renewcommand{\ReportTitle}{#1}}

\newcommand{\reporttitlepage}[1]{
  \begin{titlepage}
    {\noindent \raggedright\normalsize\sffamily \ReportAuthor}\par\vfill
    {\noindent \raggedright\Huge\sffamily \rule{\textwidth}{1pt}\\[5pt]#1\\[5pt]\rule{\textwidth}{1pt}}\par\vfill
    {\noindent \raggedright\normalsize\sffamily Version \docversion\hfill{}\docpublishingdate}\par\vfill
  \end{titlepage}
}
%%%%%%%%%%%%%%%%%%%%%%%%%%%%%%%%%%%%%%%%%%%%%%%%%%%%%%%%%%%%%%%%%%%%%%%%%%%%%%%%%%


\newcommand{\startreport}[1]
{
  \begin{document}
  \reporttitlepage{#1}
  \tableofcontents
  \specialchapter{#1}
}

\renewcommand{\thesection}{\arabic{section}}

\startreport{Meta Data Tests for Q}
\reportauthor{Ramesh Subramonian}
\newcommand{\IncludeResults}{1}
\section{Tests}
%%------- START SPEC Test 1 -----------
 \begin{definition}
 A table, \(i\), is said to be a ``dictionary'', if there is some
 field, \(f\) such that \(f.dict\_tbl\_id = i\)
 \end{definition}
 \begin{invariant}
 \label{inv_1}
 If a table is a dictionary, then it must have the following fields 
 \begin{enumerate}
 \item len with fldtype = I4 
 \item off with fldtype = I4 
 \item idx with fldtype = I4 
 \item txt with fldtype = SV 
 \end{enumerate}
 \end{invariant}
Invariant~\ref{inv_1} implemented in     Section~\ref{code_1}.
%%------- STOP  SPEC Test 1 -----------
%%------- START RSLTS Test 1 -----------
\textcolor{green}{TEST PASSED}
%%------- STOP  RSLTS Test 1 -----------
%%------- START SPEC Test 2 -----------
 \begin{invariant}
 \label{inv_2}
 If min is not defined, then min val must be 0. Same for sum, max
 \end{invariant}
Invariant~\ref{inv_2} implemented in     Section~\ref{code_2}.
%%------- STOP  SPEC Test 2 -----------
%%------- START RSLTS Test 2 -----------
\textcolor{green}{TEST PASSED}
%%------- STOP  RSLTS Test 2 -----------
%%------- START SPEC Test 3 -----------
 \begin{invariant}
 \label{inv_3}
 Relationship between min/max/sum and fldtype 
 \begin{enumerate}
 \item If min is defined, then fldtype must be one of I1, I2, I4, I8, F4, F8. 
 \item If max is defined, then fldtype must be one of I1, I2, I4, I8, F4, F8. 
 \item If sum is defined, then fldtype must be one of B, I1, I2, I4, I8, F4, F8. 
 \end{enumerate}
 \end{invariant}
Invariant~\ref{inv_3} implemented in     Section~\ref{code_3}.
%%------- STOP  SPEC Test 3 -----------
%%------- START RSLTS Test 3 -----------
\textcolor{green}{TEST PASSED}
%%------- STOP  RSLTS Test 3 -----------
%%------- START SPEC Test 4 -----------
 \begin{invariant}
 \label{inv_4}
 If field is sorted
 \begin{enumerate}
 \item fldtype must be one of I1, I2, I4, I8, F4, F8. 
 \item cannot have null values i.e., no {\tt nn} field 
 \end{enumerate}
 \end{invariant}
Invariant~\ref{inv_4} implemented in     Section~\ref{code_4}.
%%------- STOP  SPEC Test 4 -----------
%%------- START RSLTS Test 4 -----------
\textcolor{green}{TEST PASSED}
%%------- STOP  RSLTS Test 4 -----------
%%------- START SPEC Test 5 -----------
 \begin{invariant}
 \label{inv_5}
 Name of field is unique within a table 
 \end{invariant}
Invariant~\ref{inv_5} implemented in     Section~\ref{code_5}.
%%------- STOP  SPEC Test 5 -----------
%%------- START RSLTS Test 5 -----------
\textcolor{green}{TEST PASSED}
%%------- STOP  RSLTS Test 5 -----------
%%------- START SPEC Test 6 -----------
 \begin{invariant}
 \label{inv_6}
 If field is a nn field, then fldtype must be I1 or B
 \end{invariant}
Invariant~\ref{inv_6} implemented in     Section~\ref{code_6}.
%%------- STOP  SPEC Test 6 -----------
%%------- START RSLTS Test 6 -----------
\textcolor{green}{TEST PASSED}
%%------- STOP  RSLTS Test 6 -----------
%%------- START SPEC Test 7 -----------
 \begin{invariant}
 \label{inv_7}
 If field is a nn field, then nn of parent must be same as itself.
 \end{invariant}
Invariant~\ref{inv_7} implemented in     Section~\ref{code_7}.
%%------- STOP  SPEC Test 7 -----------
%%------- START RSLTS Test 7 -----------
\textcolor{green}{TEST PASSED}
%%------- STOP  RSLTS Test 7 -----------
%%------- START SPEC Test 8 -----------
 \begin{invariant}
 \label{inv_8}
 fileno is unique and positive and defined
 \end{invariant}
Invariant~\ref{inv_8} implemented in     Section~\ref{code_8}.
%%------- STOP  SPEC Test 8 -----------
%%------- START RSLTS Test 8 -----------
\textcolor{green}{TEST PASSED}
%%------- STOP  RSLTS Test 8 -----------
\section{Code Listing}
%%------- START CODE Test 1 -----------
\subsection{Invariant 1}
\label{code_1} 
\begin{verbatim}
#/usr/local/bin/bash
set -e 

q f1s1opf2 Tflds dict_tbl_id 0 '>=' x
n=`q f_to_s Tflds x sum | cut -f 1 -d ":"`
if [ $n -gt 0 ]; then 
  q delete T1
  q copy_fld Tflds dict_tbl_id x T1
  q fop T1 dict_tbl_id sortA
  q count_vals T1 dict_tbl_id '' T2 dict_tbl_id cnt
# T2 is set of tables that are dictionaries for some field
  bash check1.sh T2 "txt" "SV"
  bash check1.sh T2 "key" "I8"
  bash check1.sh T2 "len" "I4"
  bash check1.sh T2 "off" "I4"
  bash check1.sh T2 "idx" "I4"
  echo "Completed $0 in $PWD"
else 
  echo "Test $0 in $PWD not executed "
fi
q delete T1:T2
q delete Tflds x
\end{verbatim}
%%------- STOP SPEC Test 1 -----------
%%------- START CODE Test 2 -----------
\subsection{Invariant 2}
\label{code_2} 
\begin{verbatim}
#/usr/local/bin/bash
set -e 

function foo()
{
  x=$1
  q f1s1opf2 Tflds is_${x}_nn 0 '==' x
  q f1s1opf2 Tflds ${x}I8 0 '!=' y 
  q f1f2opf3 Tflds x y '&&' z
  n=`q f_to_s Tflds z sum | cut -f 1 -d ":"`
  if [ $n -ne 0 ]; then exit 1; fi 
}
foo min
foo max
foo sum


q delete Tflds x:y:z
echo "Completed $0 in $PWD"
\end{verbatim}
%%------- STOP SPEC Test 2 -----------
%%------- START CODE Test 3 -----------
\subsection{Invariant 3}
\label{code_3} 
\begin{verbatim}
#/usr/local/bin/bash
set -e 



function foo()
{
  fldtype=$1
  q regex_match lkp_fldtype txt $fldtype exact x
  idx=`q f_to_s lkp_fldtype x 'op=[get_idx]:val=[1]'`
  q f1s1opf2 Tflds fldtype $idx '==' tempf
  q f1f2opf3 Tflds tempf y '||' y
}

q s_to_f Tflds y 'op=[const]:val=[0]:fldtype=[I1]'
foo I1
foo I2
foo I4
foo I8
foo F4
foo F8

function bar() 
{
  x=$1
  q f1s1opf2 Tflds is_${x}_nn 1 '==' x
  q f1f2opf3 Tflds x y '&&!' z
  n=`q f_to_s Tflds z sum | cut -f 1 -d ":"`
  if [ $n != 0 ]; then exit 1; fi 
}
bar min
bar max

foo B
bar sum

q delete Tflds x:y:z:tempf
echo "Completed $0 in $PWD"
\end{verbatim}
%%------- STOP SPEC Test 3 -----------
%%------- START CODE Test 4 -----------
\subsection{Invariant 4}
\label{code_4} 
\begin{verbatim}
#/usr/local/bin/bash
set -e 


function foo_srt()
{
  fldtype=$1
  outfld=$2
  q regex_match lkp_fldtype txt $fldtype exact x
  idx=`q f_to_s lkp_fldtype x 'op=[get_idx]:val=[1]'`
  q f1s1opf2 Tflds fldtype $idx '==' tempf
  q f1f2opf3 Tflds tempf $outfld '||' $outfld
}


function foo_fld()
{
  fldtype=$1
  q regex_match lkp_fldtype txt $fldtype exact x
  idx=`q f_to_s lkp_fldtype x 'op=[get_idx]:val=[1]'`
  q f1s1opf2 Tflds fldtype $idx '==' tempf
  q f1f2opf3 Tflds tempf y '||' y
}

# Let x indicate fields that are sorted
q s_to_f Tflds x 'op=[const]:val=[0]:fldtype=[I1]'
foo_srt ascending x
foo_srt desceding x
foo_srt unsorted  x

# Let y indicate fields with type = one of I1, I2, I4, I8, F4, F8
q s_to_f Tflds y 'op=[const]:val=[0]:fldtype=[I1]'
foo_fld I1 y
foo_fld I2 y
foo_fld I4 y
foo_fld I8 y
foo_fld F4 y
foo_fld F8 y

# Let z = x and not y
q f1f2opf3 Tflds x y '&&!' z
n=`q f_to_s Tflds z sum | cut -f 1 -d ":"`
if [ $n != 0 ]; then exit 1; fi 

# Let y = fields that have null values
q f1s1opf2 Tflds nn_fld_id 0 '>=' y
q f1f2opf3 Tflds x y '&&' z
n=`q f_to_s Tflds z sum | cut -f 1 -d ":"`
if [ $n != 0 ]; then exit 1; fi 


q delete Tflds x:y:z:tempf
echo "Completed $0 in $PWD"
\end{verbatim}
%%------- STOP SPEC Test 4 -----------
%%------- START CODE Test 5 -----------
\subsection{Invariant 5}
\label{code_5} 
\begin{verbatim}
#/usr/local/bin/bash
set -e 


q dup_fld Tflds fk_lkp_flds x
q drop_nn_fld Tflds x  # TODO P1 Should not be needed
q f1s1opf2 Tflds tbl_id 16 '<<' y
q f1f2opf3 Tflds y x '|' z 
q fop Tflds z sortA
q count_vals Tflds z '' tempt z cnt
n1=`q get_nR Tflds`
n2=`q get_nR tempt`
if [ $n1 != $n2 ]; then echo FAILURE; exit 1; fi 
q delete Tflds x:y:z
q delete tempt
echo "Completed $0 in $PWD"
\end{verbatim}
%%------- STOP SPEC Test 5 -----------
%%------- START CODE Test 6 -----------
\subsection{Invariant 6}
\label{code_6} 
\begin{verbatim}
#/usr/local/bin/bash
set -e 

function foo_fld()
{
  fldtype=$1
  q regex_match lkp_fldtype txt $fldtype exact x
  idx=`q f_to_s lkp_fldtype x 'op=[get_idx]:val=[1]'`
  q f1s1opf2 Tflds fldtype $idx '==' tempf
  q f1f2opf3 Tflds tempf y '||' y
}


# Let x identify nn fields 
q f1s1opf2 Tflds parent_id 0 '>=' x
n=`q f_to_s Tflds x sum | cut -f 1 -d ":"`
if [ $n = 0 ]; then echo Test not executed: $0 in $PWD; exit 0; fi

# Let y identify fields that are I1 or B
q s_to_f Tflds y 'op=[const]:val=[0]:fldtype=[I1]'
foo_fld I1 y
foo_fld B  y
# Let z = x and not y
q f1f2opf3 Tflds x y '&&!' z
n=`q f_to_s Tflds z sum | cut -f 1 -d ":"`
if [ $n != 0 ]; then exit 1; fi 

q delete Tflds x:y:z:tempf
echo "Completed $0 in $PWD"
\end{verbatim}
%%------- STOP SPEC Test 6 -----------
%%------- START CODE Test 7 -----------
\subsection{Invariant 7}
\label{code_7} 
\begin{verbatim}
#/usr/local/bin/bash
set -e 


# Let x identify nn fields 
q f1s1opf2 Tflds parent_id 0 '>=' x
q delete tempt
q copy_fld Tflds id        x tempt 
q copy_fld Tflds parent_id x tempt 
q rename tempt id fld_id 
q sortf1f2 tempt parent_id fld_id A_
q is_a_in_b Tflds id tempt parent_id '' fld_id chk_nn_fld_id
q f1f2opf3 Tflds nn_fld_id chk_nn_fld_id '!=' x
n=`q f_to_s Tflds x sum | cut -f 1 -d ":"`
if [ $n != 0 ]; then echo FAILURE; exit 1; fi

q delete Tflds x:y:z:tempf
echo "Completed $0 in $PWD"
\end{verbatim}
%%------- STOP SPEC Test 7 -----------
%%------- START CODE Test 8 -----------
\subsection{Invariant 8}
\label{code_8} 
\begin{verbatim}
#/usr/local/bin/bash
set -e 


# Let x identify nn fields 
min=`q f_to_s Tflds fileno min | cut -f 1 -d ":"`
if [ $min -lt 0 ]; then echo FAILURE; exit 1; fi 
q dup_fld Tflds fileno x
q fop Tflds x sortA
q delete tempt
q count_vals Tflds x '' tempt fileno cnt
nR1=`q get_nR Tflds`
nR2=`q get_nR tempt`
if [ $nR1 -ne $nR2 ]; then echo FAILURE; exit 1; fi 

q delete Tflds x
q delete tempt
echo "Completed $0 in $PWD"
\end{verbatim}
%%------- STOP SPEC Test 8 -----------
\end{document}
